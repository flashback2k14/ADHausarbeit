Zur Lösung des Problems von kostenminimalen Flüssen in Netzwerken verwendet
man den $Busacker-Gowen-Algorithmus$. Dieser Algorithmus ist eine 
Weiterentwicklung der Algorithmen von $Ford-Fulkerson$ und $Edmonds-Karp$. 
Das Problem dieser Algorithmen ist, dass diese nicht kostenminimal arbeiten. 
Der Algorithmus von $Ford-Fulkerson$ kommt zum Einsatz zur Bestimmung eines 
maximalen Flusses zwischen einer Quelle $s$ und einer Senke $t$. Zu Beginn 
wird der Algorithmus mit einem bestehenden Fluss, z.B.: Nullfluss 
initialisiert. Anschließend wird ein Weg von der Quelle $s$ zur Senke $t$ 
gesucht auf dem alle Kanten positive Kapazitäten besitzen. Im nächsten 
Schritt wird der Fluss um die kleinste Kapazität auf diesem Pfad $p$
erhöht. Dieser Vorgang erfolgt solange bis kein Pfad im Netzwerk $N$ mit 
positiven Kanten existiert. Die Laufzeit beträgt $O(|V(G)| + |E(G)|)$ 
\cite{algo, optiv}\\

\lstset{caption={Ford-Fulkerson-Algorithmus}, language={Python}}
\begin{lstlisting}
Ford-Fulkerson(G,s,t)
    for jede Kante (u,v) in G.E 
        (u,v).f = 0
    while es existiert ein Pfad p von s nach t im Restnetzwerk G von f
       c von f = min {c von f(u,v) : (u,v) gehoert zu p}
       for jede Kante (u,v) von p
            if (u,v) in G.E
                (u,v).f = (u,v) + c von f(p)
            else
                (v,u).f = (v,u).f - c von f(p)
\end{lstlisting}

Im Gegenteil dazu implementiert der Algorithmus von $Edmonds-Karp$ zur 
Berechnung des Pfads $p$ eine Breitensuche. Die Breitensuche wählt für den 
Pfad $p$ einen kürzesten Pfad von $s$ nach $t$ aus, wobei jede Kante das 
Gewicht / Distanz 1 erhält. Daraus folgt eine Laufzeitverbesserung auf 
$O(|V(G)| + |E(G)|^2)$ \cite{algo, optiv}

Der $Busacker-Gowen-Algorithmus$ verfolgt die Idee einen kostenminimalen 
Fluss von $s$ nach $t$ zu konstruieren. Der Start bildet entweder ein 
Nullfluss oder ein gegebener Fluss. Das Ziel ist das Finden eines vorgebenen 
bzw. maximalen Flusses, d.h. erreichen der Maximalkapazität. Dies wird 
erreicht in dem man bei jedem Durchgang einen flussvergrößernden Pfad $p$ im 
Inkrementnetzwerk von $s$ nach $t$ sucht, der unter allen $s-t-Pfaden$ die 
kleinsten Kosten verursacht. Der Algorithmus lässt sich in 2 Schritte 
unterteilen:
\begin{enumerate}
 \item Bestimmung des kürzesten Weges von $s$ nach $t$
 \item Vergrößerung Fluss und Neuberechnung Inkrementdigraph bis vorgegebene 
 Wert oder Maximalkapazität des Netzwerkes $N$ erreicht ist. 
\end{enumerate}
Im ersten Schritt konstruiert man einen bewerteten Inkrementdigraph aus dem 
Netzwerk $N$ für jede gerichtete Kante $(i,j)$ aus $N$ mit $f_{ij} < l_{i,j}$
mit der Bewertung $c_{ij}$. Des Weiteren wir für jede gerichtete Kante 
$(i,j)$ von $N$ mit $f_{ij} > 0$ eine umgekehrt gerichtete Kante $(j,i)$ mit 
der Bewertung $-c_{ij}$ dem Graphen hinzugefügt. Im Anschluss wird mittels 
eines kürzeste-Wege-Algorithmus ein $s-t-Pfad$ ermittelt, welcher 
gleichzeitig ein flussvergrößernder und kostenminimaler Weg in $N$ ist. Der 
Schritt 2 wird solange durchgeführt bis die vorgegebene Flussstärke erreicht
oder bis keine weitere Flussvergrößerung möglich ist. Für ein 
Inkrementnetzwerk gelten
folgende Bedienungen:
\begin{enumerate}
 \item $E(\Phi)=E^+(\Phi)\cup E^-(\Phi)$
 \item $E^+(\Phi)=\{<i,j>|<i,j> \in E,\Phi_{ij} < \kappa_{ij}\}$
 \item $E^-(\Phi)=\{<j,i>|<i,j> \in E,\Phi_{ij} > 0\}$
 \item $\kappa_{ij}(\Phi)=\biggl[\begin{array}{ll}\kappa_{ij}-\Phi_{ij} falls
 <i,j> \in E^+(\Phi) \\ \Phi_{ji} falls <i,j> \in E^-(\Phi) \end{array}$
 \item $c_{ij}(\Phi)=\biggl[\begin{array}{ll}c_{ij}falls<i,j> \in E^+(\Phi)
  \\ -c_{ji}falls<i,j> \in E^-(\Phi)\end{array}$
\end{enumerate}

\lstset{caption={Busacker-Gowen-Algorithmus}, language={Python}}
\begin{lstlisting}
Busacker-Gowen(G,s,t)
    for jede Kante (u,v) in G.E 
        (u,v).f = 0
    while es existiert ein Pfad p von s nach t im Restnetzwerk G von f
        ???
    
\end{lstlisting}

Die Laufzeit des Algorithmus beträgt $O(m * l_\lambda * log_n)$ bzw. 
$O(m * l^{max} * log_m)$ mit vorgebener Flussstärke $l_\lambda$ bzw. 
Maximalkapazität $l^{max}$. \cite{kripfganz, optiv, tudortmund}