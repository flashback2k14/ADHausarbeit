Das naheliegendste Beispiel f\"ur ein Flussproblem ist eine 
Firma die Waren in einer Fabrik $s$ produziert und 
kontinuierlich zu ihrer Verkaufstelle $t$ transportieren m\"ochte. 
Dabei stehen verschiedene Zwischenlager 
(Knoten $v \in V \setminus \{s,t\}$)  zur Verf\"ugung und eine 
maximale Menge $\kappa(u,v)$ kann in einem Tag von einem 
Zwischenlager (oder Fabrik) zu einem anderen Zwischenlager 
(oder Verkaufstelle) transportiert werden. Unterschiedliche Kosten
f\"ur Transportwege treten bei dem Benutzten von verschiedenen 
Transportmitteln, z.B. LKW, Zug oder Flugzeug, auf. 

Als weiteres Beispiel sei die Paarung von m\"oglichen Partner 
einer Partnerb\"orse genannt. Gegeben sei eine Menge $A$ von 
Frauen und eine Menge $B$ von M\"annern. 
Durch soziologische Pr\"aferenzen von Frauen und  M\"anner 
ergeben sich potentielle Paarungen $E \subseteq A \times B$
zwischen den beiden Gruppen. Der Betreiber einer Partnerb\"orse 
m\"ochte nun m\"oglichst viele
solcher potentiellen Paare mit einander zusammenbringen.
Dabei soll kein Partner in mehreren Paarungen auftreten. 
Das algorithmische Problem hierbei das Finden eines 
\emph{maximalen Matchings}, d.h. eine Kantenmenge 
$M \subset E$ des bipartiten Graphen $G = (A \cup B, E)$ mit 
maximaler Gr\"o\ss e, so dass der Teilgraph von $G$ der nur 
die Kanten $M$ besitzt Maximalgrad $1$ hat. Ein maximales Matching
kann gefunden werden in dem zu $G$ eine k\"unstliche Quelle $s$, 
die eine gerichtete Kante zu jedem Knoten aus $A$ erh\"alt, 
und eine Senke $t$, zu der eine gerichtete Kante von jedem 
Knoten in $B$ f\"uhrt, hinzugef\"ugt wird. Weiterhin wird jede
Kante in $E$ von $A$ nach $B$ gerichtet und wir geben jeder 
Kante Kapazit\"at eins. Es ist leicht zu sehen, 
dass ein Maximaler Fluss in diesem Graphen einem Maximalen 
Matching in $G$ entspricht: $M$ entspricht allen Kanten von $A$
nach $B$ mit Fluss $1$. \cite{schoening, algo}

%\footnote{
%Es ist zu bermerken das dieses Beispiel in Bezug auf Diversit\"at
%ungeeignet ist und dass in der realen Welt gleichgeschlechtliche 
%Beziehungen existieren. Da jedoch die hier erl\"auterten 
%Algorithmen nur mit bipartiten Graphe korrekt funktionieren wurde
%dieses ansonsten sehr anschauliche Beispiel gew\"ahlt.
%}.

%Weitere Anwendungsgebiete sind mit der aus der Graphentheorie 
%bekannten \"Aquivalenz von maximalen F\"ussen und minimalen S
%Schnitten zu entdecken. Ein \emph{minmaler Schnitte} f\"ur 
%zwei Knoten $s,t$ eines Graphen $G = (V,E)$ is eine Teilmenge
% $S \subseteq E$ der Kanten mit minimaler Gr\"o\ss e, 
%so dass nach Entfernen von $S$ aus $G$ kein Weg von $s$ nach $t$ 
%mehr existiert. Wenn wir uns $s$ als das Versorgungsdepot 
%und $t$ als Frontlinie eines milit\"arischen Gegners vorstellen, 
%und $G$ das zugrunde liegende Transportnetzwerk ist, kann 
%ein minimaler Schnitt in $G$ als der effizienteste Angriffsplan
%interpretiert werden, um die Lieferung von Nachsch\"uben aus 
%dem Versorgungsdepot and die Front zu unterbinden.