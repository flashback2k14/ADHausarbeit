Ein \emph{Netzwerk} ist ein gerichteter Graph $G=(V,E)$ mit 
einer Kapazitätsfunktion $\kappa: V \times V \rightarrow \N$ 
die jeder Kante eine maximale Kapazität zuordnet und zwei 
dedizierten Knoten $s,t \in V$. Der Knoten $s$ ist die 
\emph{Quelle} und besitzt nur ausgehende Kanten. Der Knoten 
$t$ wird als \emph{Senke} bezeichnet und besitzt nur eingehende 
Kanten. Ein \emph{Fluss} ist eine Funktion 
$f: V \times V \rightarrow \N$ die jeder Kante eine Anzahl 
an Einheiten zuordnet (Kapazität), die über die Kante 
transportiert werden können. Wir nennen einen Fluss 
\emph{zulässig} wenn für alle Kanten. $(u,v) \in E$ folgendes gilt
\begin{enumerate}
 \item $f(u,v) \leq \kappa(u,v)$, d.h.~der Fluss darf die 
 maximale Kapazität der Kanten nicht überschreiten
 \item für alle Knoten $v \in V\setminus \{s,t\}$ ist
 $\sum_{(u,v)\in E}f(u,v) = \sum_{(v,u)\in E}f(v,u)$. 
\end{enumerate}
Die zweite Bedingung ist auch als Kirchhoffsches Gesetz bekannt. 
Der \emph{Wert} eines Fluss $|f|$ ist die Summe der von $s$ 
ausgehenden Einheiten, d.h., $|f| = \sum_{(s,v) \in E}f(s,v)$. 
Das \emph{Maximale Fluss Problem} fragt nach dem zulässigen
Fluss $f$ für ein gegebenes Netzwerk $G$, der den maximalen 
Wert hat. Zusätzlich zu den Kapazitäten der Kanten kann eine
Gewichtsfunktions $c: V \times V \rightarrow \N$ gegeben sein, 
so dass das transportieren einer Einheit über eine Kante $(u,v)$
Kosten $c(u,v)$ verursacht. Dies führt zur Definition des
\emph{Gewichtes} $w(f)$ eines Flusses als die Summe der 
Produkte des Gewichtes und des Flusses der Kanten. Die Eingabe 
für das \emph{Kostenminimale Fluss Problem} ist ein Netzwerk $G$ 
mit einer Kostenfunktion $c$ und einem maximalen Wert $F$. 
Gesucht wird nach einem Fluss in $G$ mit Wert mindestens $F$ 
der die minimalen Kosten unter allen solchen Flüssen besitzt.
Das Maximale Fluss Problem entspricht hier dem Kostenminimalem 
Fluss Problem mit dem $c(u,v) = 1$ für alle $(u,v) \in E$ 
und $F = \infty$. \cite{schoening, kripfganz, algo}