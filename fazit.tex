Zusammenfassend erfüllt das Programm die Aufgabe des zeilenweise Findens von
kostenminimalen Flüssen (gerichteten Kanten) in einem Netzwerk mit einer gegebenen 
Maximalkapazität und gegebenen Kosten mithilfe eines Inkrementnetzwerkes. Das Projekt 
wurde auf Basis von Java 7 mit der Entwicklungsumgebung Eclipse 4.4.1 erstellt. Das 
Projekt lässt sich durch folgende Kommandos in der Eingabeaufforderung ausführen
\begin{enumerate}
\item java -jar kostenminimalerfluss.jar $1$
\item java -jar kostenminimalerfluss.jar $2$ 
\item java -jar kostenminimalerfluss.jar $/Pfad/zu/Daten.txt$ $Quelle$ $Senke$ 
$Flussstaerke$ 
\end{enumerate}
Im ersten Punkt wird das Programm mit den Daten für die Hausarbeit, im zweiten Punkt
mit den Daten aus der Übung 6 Aufgabe 2 ausgeführt. Des Weiteren ist es mögliche das
Programm mittels Angabe einer Textdatei, der Quelle, der Senke und der Flussstärke 
auszuführen. In der .zip Datei (cs13\_kloppe\_sebastian.zip) auf dem FTP-Server der 
Berufsakademie ist der gesamte Quellcode des Programms zu finden.