\documentclass[a4paper, 10pt]{article}
%--------------------------------------
% Ränder + Abstände
%--------------------------------------
\usepackage[left=30mm,right=20mm,top=25mm,headsep=15mm]{geometry}
\usepackage[onehalfspacing]{setspace}
%--------------------------------------
% encoding
%--------------------------------------
\usepackage[utf8]{inputenc}
\usepackage[T1]{fontenc}
%--------------------------------------
% German-specific commands
%--------------------------------------
\usepackage[ngerman]{babel}
%--------------------------------------
% Mathe
%--------------------------------------
\usepackage[noend, linesnumbered, boxed]{algorithm2e}
\usepackage{amsmath, amsthm, amssymb}
\newtheorem{lemma}{Lemma}
\DeclareMathOperator{\N}{\mathbb{N}}
%--------------------------------------
% Pseudocode
%--------------------------------------
\usepackage{listings}
\lstset{
    language=VBScript, 
    frame=single, 
    breaklines=true,
    numbers=left, 
    numbersep=10pt, 
    basicstyle=\footnotesize, 
    captionpos=b, 
    morekeywords={end,do,repeat,print,output}, 
    xleftmargin=20pt, 
    framexleftmargin=0pt, 
    framexrightmargin=0pt
}
%--------------------------------------
% Quellenverzeichnis 
%--------------------------------------
\usepackage{csquotes}
\usepackage[backend=biber,defernumbers,sorting=none,firstinits=true]{biblatex}
\addbibresource{biblo.bib}
%--------------------------------------
% Dokument
%--------------------------------------
\begin{document}
%--------------------------------------
% Deckblatt
%--------------------------------------
\thispagestyle{empty}
\begin{center}
\Large{Berufsakademie Sachsen}
\end{center}
 
\begin{center}
\Large{Staatliche Studienakademie Leipzig}
\end{center}
\hspace{4cm}

\begin{center}
\textbf{\LARGE{Hausarbeit}}\\
\end{center}

\begin{center}
\textbf{im Studiengang Informatik}\\
\end{center}
\hspace{4cm}

\begin{flushleft}
\begin{tabular}{l p{10pt} p{290pt}}
%\textbf{im Studiengang Informatik}\\
\\
\textbf{Thema:} & & Kostenminimale F\"usse in Netzwerken - 12B\\
& & \\
& & \\
& & \\
\textbf{eingereicht von:} & & Sebastian Kloppe \\
& &                           Einertstraße 10\\
& &                           04315 Leipzig\\
& & \\
\textbf{Seminargruppe:} & & CS13-2 \\
\textbf{Matrikelnummer:} & & 5000349\\
& & \\
& & \\
& & \\
& & \\
& & \\
& & \\
& & \\
& & \\
& & \\
& & \\
& & \\
& & \\
\textbf{eingereicht am:} & & 27. M\"arz 2015 in Leipzig\\
\end{tabular}
\end{flushleft}
%--------------------------------------
% Inhaltsverzeichnis
%--------------------------------------
\newpage
\pagenumbering{Roman}
\tableofcontents
\lstlistoflistings
%--------------------------------------
% Einleitung
%--------------------------------------
\clearpage 
\pagenumbering{arabic} 
\section{Einleitung}
Ein \emph{Netzwerk} ist ein gerichteter Graph $G=(V,E)$ mit 
einer 
Kapazit\"atsfunktion $\kappa: V \times V \rightarrow \N$ 
die jeder Kante 
eine maximale Kapazit\"at zuordnet und zwei dedizierten Knoten
$s,t \in V$. Der Knoten $s$ ist die \emph{Quelle} und 
besitzt nur ausgehende Kanten. Der Knoten $t$ wird als 
\emph{Senke} bezeichnet und besitzt nur eingehende Kanten. 
Ein \emph{Fluss} ist eine Funktion $f: V \times V \rightarrow \N$ 
die jeder Kante 
eine Anzahl an Einheiten zuordnet (Kapazit\"at), die \"uber die Kante 
transportiert werden. Wir nennen einen Fluss \emph{zul\"assig}
wenn f\"ur alle Kanten. $(u,v) \in E$ folgendes gilt
\begin{enumerate}
 \item $f(u,v) \leq \kappa(u,v)$, d.h.~der Fluss darf die 
 maximale Kapazit\"at der Kanten nicht \"uberschreiten
 \item f\"ur alle Knoten $v \in V\setminus \{s,t\}$ ist
 $\sum_{(u,v)\in E}f(u,v) = \sum_{(v,u)\in E}f(v,u)$. 
\end{enumerate}
Die zweite Bedingung ist auch als Kirchhoffsches Gesetz bekannt. 
Der \emph{Wert} eines Fluss $|f|$ ist die Summe der 
von $s$ ausgehenden Einheiten, d.h., 
$|f| = \sum_{(s,v) \in E}f(s,v)$. 
Das \emph{Maximale Fluss Problem} fragt nach dem zul\"assigen
Fluss $f$ f\"ur ein gegebenes Netzwerk $G$, der den maximalen 
Wert hat. Zus\"atzlich zu den Kapazit\"aten der Kanten kann 
eine Gewichtsfunktions $c: V \times V \rightarrow \N$ gegeben 
sein, so dass das transportieren einer Einheit \"uber eine Kante $(u,v)$
Kosten $c(u,v)$ verursacht. Dies f\"uhrt zur Definition des
\emph{Gewichtes} $w(f)$ eines Flusses als die Summe der 
Produkte des Gewichtes und des Flusses der Kanten. 
Die Eingabe f\"ur das \emph{Kostenminimale Fluss Problem} ist 
ein Netzwerk $G$ mit einer Kostenfunktion $c$ und einem maximalen 
Wert $F$. Gesucht wird nach einem Fluss in $G$ mit Wert mindestens
$F$ der die minimalen Kosten unter allen solchen Fl\"ussen besitzt.
Das Maximale Fluss Problem entspricht hier dem
Kostenminimalem Fluss Problem mit dem $c(u,v) = 1$ f\"ur alle 
$(u,v) \in E$ und $F = \infty$. \cite{schoening, kripfganz, algo}
%--------------------------------------
% Motivation
%--------------------------------------
\section{Motivation}
Das naheliegendste Beispiel f\"ur ein Flussproblem ist eine 
Firma die Waren in einer Fabrik $s$ produziert und 
kontinuierlich zu ihrer Verkaufstelle $t$ transportieren m\"ochte. 
Dabei stehen verschiedene Zwischenlager 
(Knoten $v \in V \setminus \{s,t\}$)  zur Verf\"ugung und eine 
maximale Menge $\kappa(u,v)$ kann in einem Tag von einem 
Zwischenlager (oder Fabrik) zu einem anderen Zwischenlager 
(oder Verkaufstelle) transportiert werden. Unterschiedliche Kosten
f\"ur Transportwege treten bei dem Benutzten von verschiedenen 
Transportmitteln, z.B. LKW, Zug oder Flugzeug, auf. 

Als weiteres Beispiel sei die Paarung von m\"oglichen Partner 
einer Partnerb\"orse genannt. Gegeben sei eine Menge $A$ von 
Frauen und eine Menge $B$ von M\"annern. 
Durch soziologische Pr\"aferenzen von Frauen und  M\"anner 
ergeben sich potentielle Paarungen $E \subseteq A \times B$
zwischen den beiden Gruppen. Der Betreiber einer Partnerb\"orse 
m\"ochte nun m\"oglichst viele
solcher potentiellen Paare mit einander zusammenbringen.
Dabei soll kein Partner in mehreren Paarungen auftreten. 
Das algorithmische Problem hierbei das Finden eines 
\emph{maximalen Matchings}, d.h. eine Kantenmenge 
$M \subset E$ des bipartiten Graphen $G = (A \cup B, E)$ mit 
maximaler Gr\"o\ss e, so dass der Teilgraph von $G$ der nur 
die Kanten $M$ besitzt Maximalgrad $1$ hat. Ein maximales Matching
kann gefunden werden in dem zu $G$ eine k\"unstliche Quelle $s$, 
die eine gerichtete Kante zu jedem Knoten aus $A$ erh\"alt, 
und eine Senke $t$, zu der eine gerichtete Kante von jedem 
Knoten in $B$ f\"uhrt, hinzugef\"ugt wird. Weiterhin wird jede
Kante in $E$ von $A$ nach $B$ gerichtet und wir geben jeder 
Kante Kapazit\"at eins. Es ist leicht zu sehen, 
dass ein Maximaler Fluss in diesem Graphen einem Maximalen 
Matching in $G$ entspricht: $M$ entspricht allen Kanten von $A$
nach $B$ mit Fluss $1$. \cite{schoening, algo}

%\footnote{
%Es ist zu bermerken das dieses Beispiel in Bezug auf Diversit\"at
%ungeeignet ist und dass in der realen Welt gleichgeschlechtliche 
%Beziehungen existieren. Da jedoch die hier erl\"auterten 
%Algorithmen nur mit bipartiten Graphe korrekt funktionieren wurde
%dieses ansonsten sehr anschauliche Beispiel gew\"ahlt.
%}.

%Weitere Anwendungsgebiete sind mit der aus der Graphentheorie 
%bekannten \"Aquivalenz von maximalen F\"ussen und minimalen S
%Schnitten zu entdecken. Ein \emph{minmaler Schnitte} f\"ur 
%zwei Knoten $s,t$ eines Graphen $G = (V,E)$ is eine Teilmenge
% $S \subseteq E$ der Kanten mit minimaler Gr\"o\ss e, 
%so dass nach Entfernen von $S$ aus $G$ kein Weg von $s$ nach $t$ 
%mehr existiert. Wenn wir uns $s$ als das Versorgungsdepot 
%und $t$ als Frontlinie eines milit\"arischen Gegners vorstellen, 
%und $G$ das zugrunde liegende Transportnetzwerk ist, kann 
%ein minimaler Schnitt in $G$ als der effizienteste Angriffsplan
%interpretiert werden, um die Lieferung von Nachsch\"uben aus 
%dem Versorgungsdepot and die Front zu unterbinden.
%--------------------------------------
% Beschreibung
%--------------------------------------
\section{Beschreibung \& Komplexität}
Zur Lösung des Problems von kostenminimalen Flüssen in Netzwerken verwendet
man den $Busacker-Gowen-Algorithmus$. Dieser Algorithmus ist eine 
Weiterentwicklung der Algorithmen von $Ford-Fulkerson$ und $Edmonds-Karp$. 
Das Problem dieser Algorithmen ist, dass diese nicht kostenminimal arbeiten. 
Der Algorithmus von $Ford-Fulkerson$ kommt zum Einsatz zur Bestimmung eines 
maximalen Flusses zwischen einer Quelle $s$ und einer Senke $t$. 

\lstset{caption={Ford-Fulkerson-Algorithmus}, language={VBScript}}
\begin{lstlisting}
FordFulkerson(G,s,t)
    for jede Kante (u,v) in G.E 
        (u,v).f = 0
    end for
    
    while es existiert ein Pfad p von s nach t im Restnetzwerk G von f do
       c von f = min {c von f(u,v) : (u,v) gehoert zu p}
       for jede Kante (u,v) von p
            if (u,v) in G.E
                (u,v).f = (u,v) + c von f(p)
            else
                (v,u).f = (v,u).f - c von f(p)
            end if
        end for
    end while
\end{lstlisting}

Zu Beginn wird der Algorithmus mit einem bestehenden Fluss, z.B.: Nullfluss 
initialisiert. Anschließend wird ein Weg von der Quelle $s$ zur Senke $t$ 
gesucht auf dem alle Kanten positive Kapazitäten besitzen. Im nächsten 
Schritt wird der Fluss um die kleinste Kapazität auf diesem Pfad $p$
erhöht. Dieser Vorgang erfolgt solange bis kein Pfad im Netzwerk $N$ mit 
positiven Kanten existiert. Die Laufzeit beträgt $O(|V(G)| + |E(G)|)$ ,
weil \textbf{TODO: ???}. \cite{algo, optiv}\\

Im Gegenteil dazu implementiert der Algorithmus von $Edmonds-Karp$ zur 
Berechnung des Pfads $p$ eine Breitensuche. 

\lstset{caption={Edmonds-Karp-Algorithmus}, language={VBScript}}
\begin{lstlisting}
EdmondsKarp(G,s,t)
    for jede Kante (u,v) in G.E 
        (u,v).f = 0
    end for
    
    while es existiert ein Pfad p von s nach t im Restnetzwerk G von f do
        1. sei p ein Pfad von s nach t in G von f mit minimaler Anzahl von Konten
        2. Vergroessere f mit p
        3. Aktualisiere G von f
    end while
\end{lstlisting}

Die Breitensuche wählt für den 
Pfad $p$ einen kürzesten Pfad von $s$ nach $t$ aus, wobei jede Kante das 
Gewicht / Distanz 1 erhält. Daraus folgt eine Laufzeitverbesserung auf 
$O(|V(G)| + |E(G)|^2)$, weil \textbf{TODO: ???}. \cite{algo, optiv, edmkarp} Der 
$Busacker-Gowen-Algorithmus$ verfolgt die Idee einen kostenminimalen 
Fluss von $s$ nach $t$ zu konstruieren. 

\lstset{caption={Busacker-Gowen-Algorithmus}, language={VBScript}}
\begin{lstlisting}
BusackerGowen(G,s,t)
    for jede Kante (u,v) in G.E 
        (u,v).f = 0
    end for
    
    while es existiert ein Pfad p von s nach t im Restnetzwerk G von f do
        1. Bestimmung des kuerzesten Weges von s nach t mit Bellman-Ford-Algorithmus
        2. Vergroesserung des Flusses
        3. Neuberechnung Inkrementdigraph bis vorgegebene Wert oder Maximalkapazitaet
        des Netzwerkes N erreicht ist
    end while
\end{lstlisting}

\lstset{caption={Bellman-Ford-Algorithmus}, language={VBScript}}
\begin{lstlisting}
BellmanFord(G,s)
    for jedes v aus V                   
        Distanz(v) := unendlich, Vorgaenger(v) := keiner
        Distanz(s) := 0
    end for
    
    while n - 1               
        for jede Kante (u,v) in G.E
            if Distanz(u) + Gewicht(u,v) < Distanz(v)
            then
              1. Distanz(v) := Distanz(u) + Gewicht(u,v)
              2. Vorgaenger(v) := u
            end if
        end for
    end while
        
    for jede Kante (u,v) in G.E                
        if Distanz(u) + Gewicht(u,v) < Distanz(v)
        then
            STOP print "Es gibt einen Zyklus negativen Gewichtes."
        end if
    end for

    output Distanz
\end{lstlisting}

Der Start bildet entweder ein Nullfluss oder ein gegebener Fluss. Das Ziel 
ist das Finden eines vorgebenen bzw. maximalen Flusses, d.h. erreichen der 
Maximalkapazität. Dies wird erreicht in dem man bei jedem Durchgang einen 
flussvergrößernden Pfad $p$ im Inkrementnetzwerk von $s$ nach $t$ sucht, der
unter allen $s-t-Pfaden$ die kleinsten Kosten verursacht. Der Algorithmus 
lässt sich in 2 Schritte unterteilen. Im ersten Schritt konstruiert man einen 
bewerteten Inkrementdigraph aus dem Netzwerk $N$ für jede gerichtete Kante $(i,j)$ aus
$N$ mit $f_{ij} < l_{i,j}$ mit der Bewertung $c_{ij}$. Des Weiteren wir für jede 
gerichtete Kante $(i,j)$ von $N$ mit $f_{ij} > 0$ eine umgekehrt gerichtete Kante 
$(j,i)$ mit der Bewertung $-c_{ij}$ dem Graphen hinzugefügt. Im Anschluss 
wird mittels eines kürzeste-Wege-Algorithmus ein $s-t-Pfad$ ermittelt, 
welcher gleichzeitig ein flussvergrößernder und kostenminimaler Weg in $N$ 
ist. In diesem Fall kommt der Algorithmus von $Bellman-Ford$ zum Einsatz. Dieser 
findet für jedem Knoten $x$ einen kürzesten Weg von $s$ nach $x$ mit maximal $k$ 
Kanten. Darausfolgt, dass ein Weg ohne Zyklen maximal $n$ Knoten und $n-1$ Kanten 
besitzt. Ein Zyklus entsteht bei einem Weg mit negativem Weg. Der Schritt 2 wird 
solange durchgeführt bis die vorgegebene Flussstärke erreicht oder bis keine weitere 
Flussvergrößerung möglich ist. Für ein Inkrementnetzwerk gelten folgende Bedienungen:
\begin{enumerate}
 \item $E(\Phi)=E^+(\Phi)\cup E^-(\Phi)$
 \item $E^+(\Phi)=\{<i,j>|<i,j> \in E,\Phi_{ij} < \kappa_{ij}\}$
 \item $E^-(\Phi)=\{<j,i>|<i,j> \in E,\Phi_{ij} > 0\}$
 \item $\kappa_{ij}(\Phi)=\biggl[\begin{array}{ll}\kappa_{ij}-\Phi_{ij} 
 falls
 <i,j> \in E^+(\Phi) \\ \Phi_{ji} falls <i,j> \in E^-(\Phi) \end{array}$
 \item $c_{ij}(\Phi)=\biggl[\begin{array}{ll}c_{ij}falls<i,j> \in E^+(\Phi)
  \\ -c_{ji}falls<i,j> \in E^-(\Phi)\end{array}$
\end{enumerate}
\textbf{TODO: Aufzählung genauer beschreiben!?}

Die Laufzeit des Algorithmus beträgt $O(|E(G)| * l_\lambda * log_{|V(G)|})$ 
bzw. $O(|E(G)| * l^{max} * log_{|E(G)|})$ mit vorgebener Flussstärke 
$l_\lambda$ bzw. Maximalkapazität $l^{max}$, weil \textbf{TODO: ???}. 
\cite{kripfganz, optiv, tudortmund, bellford}
%--------------------------------------
% Ergebnis
%--------------------------------------
\section{Ergebnis}
\input{ergebnis}
%--------------------------------------
% Fazit
%--------------------------------------
\section{Fazit}
Zusammenfassend erfüllt das Programm die Aufgabe des zeilenweise Findens von
kostenminimalen Flüssen (gerichteten Kanten) in einem Netzwerk mit einer gegebenen 
Maximalkapazität und gegebenen Kosten mithilfe eines Inkrementnetzwerkes. Das Projekt 
wurde auf Basis von Java 7 mit der Entwicklungsumgebung Eclipse 4.4.1 erstellt. Das 
Projekt lässt sich durch folgende Kommandos in der Eingabeaufforderung ausführen
\begin{enumerate}
\item java -jar kostenminimalerfluss.jar $1$
\item java -jar kostenminimalerfluss.jar $2$ 
\item java -jar kostenminimalerfluss.jar $/Pfad/zu/Daten.txt$ $Quelle$ $Senke$ 
$Flussstaerke$ 
\end{enumerate}
Im ersten Punkt wird das Programm mit den Daten für die Hausarbeit, im zweiten Punkt
mit den Daten aus der Übung 6 Aufgabe 2 ausgeführt. Des Weiteren ist es mögliche das
Programm mittels Angabe einer Textdatei, der Quelle, der Senke und der Flussstärke 
auszuführen. In der .zip Datei (cs13\_kloppe\_sebastian.zip) auf dem FTP-Server der 
Berufsakademie ist der gesamte Quellcode des Programms zu finden.
%--------------------------------------
% Quellen
%--------------------------------------
\newpage
\section{Quellenverzeichnis}
\printbibliography
%--------------------------------------
% Selbstständigkeit
%--------------------------------------
\newpage
\section{Selbstständigkeitserklärung}
\normalsize{Ich versichere, dass ich die vorliegende Arbeit ohne fremde Hilfe selbstständig verfasst und nur die angegebenen Quellen und Hilfsmittel benutzt habe. Wörtlich 
oder dem Sinn nach aus anderen
Werken entnommene Stellen sind unter Angabe der Quellen kenntlich gemacht. Die Arbeit wurde bisher in gleicher oder ähnlicher Form weder 
veröffentlicht, noch 
einer anderen Prüfungsbehörde vorgelegt.
\newline
\newline
\newline
\newline
Leipzig, den \today}

\end{document}