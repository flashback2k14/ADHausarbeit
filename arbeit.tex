\documentclass[a4paper]{article}
\usepackage{fullpage}


\usepackage[noend, linesnumbered, boxed]{algorithm2e}
\usepackage{amsmath, amsthm, amssymb}
\newtheorem{lemma}{Lemma}


\DeclareMathOperator{\N}{\mathbb{N}}

%\input{custom-preambl}


\title{Hausarbeit \"uber Kostenminimale F\"usse in Netzwerken}
\author{Sebastian Kloppe}

\begin{document}
\maketitle
\section{Einleitung}
Ein \emph{Netzwerk} ist ein gerichteter Graph $G=(V,E)$ mit 
einer 
Kapazit\"atsfunktion $\kappa: V \times V \rightarrow \N$ 
die jeder Kante 
eine maximale Kapazit\"at zuordnet und zwei dedizierten Knoten
$s,t \in V$. Der Knoten $s$ ist die \emph{Quelle} und 
besitzt nur ausgehende Kante und der Knoten $t$ is die 
\emph{Senke} und besitzt nur eingehende Kanten. 
Ein \emph{Fluss} ist eine Funktion $f: V \times V \rightarrow \N$ 
die jedem Kante 
eine Anzahl an Einheiten zuordnet, die \"uber die Kante 
transportiert werden. Wir nennen einen Fluss \emph{zul\"assig}
wenn f\"ur alle Kanten. $(u,v) \in E$ folgendes gilt
\begin{enumerate}
 \item $f(u,v) \leq \kappa(u,v)$, d.h.~der Fluss darf die 
 maximale Kapazit\"at der Kanten nicht \"uberschreiten
 \item f\"ur alle Knoten $v \in V\setminus \{s,t\}$ ist
 $\sum_{(u,v)\in E}f(u,v) = \sum_{(v,u)\in E}f(v,u)$. 
\end{enumerate}
Die zweite Bedingung ist auch als Kirchhoffsches Gesetz bekannt. 
Der \emph{Wert} eines Fluss $|f|$ ist die Summe der 
von $s$ ausgehenden Einheiten, d.h., 
$|f| = \sum_{(s,v) \in E}f(s,v)$. 
Das \emph{Maximaler Fluss Problem} fragt nach dem zul\"assigen
Fluss $f$ f\"ur ein gegebenes Netzwerk $G$, der den maximalen 
Wert hat. Zus\"atzlich zu den Kapazit\"aten der Kanten kann 
eine Gewichtsfunktions $c: V \times V \rightarrow \N$ gegeben 
sein und das transportieren einer Einheit \"uber eine Kante $(u,v)$
verursacht Kosten $c(u,v)$. Dies f\"uhrt zur Definition des
\emph{Gewichtes} $w(f)$ eines Flusses als die Summe der 
Produkte des Gewichtes und des Flusses der Kanten. 
Die Eingabe f\"ur das \emph{Kostenminimale Fluss Problem} ist 
ein Netzwerk $G$ mit einer Kostenfunktion $c$ und einem maximalen 
Wert $F$. Gesucht wird nach einem Fluss in $G$ mit Wert mindestens
$F$ der die minimalen Kosten unter allen solchen Fl\"ussen besitzt.
Der Maximaler Fluss Problem entspricht hier dem
Kostenminimalem Fluss Problem mit dem $c(u,v) = 1$ f\"ur alle 
$(u,v) \in E$ und $F = \infty$.

\section{Motivation}
Das naheliegendste Beispiel f\"ur ein Flussproblem ist eine 
Firma die Waren in einer Fabrik $s$ produziert und 
kontinuierlich zu ihrer Verkaufstelle $t$ transportieren m\"ochte. 
Dabei stehen verschiedene Zwischenlager 
(Knoten $v \in V \setminus \{s,t\}$  zur Ver\"ugung und eine 
maximale Menge $\kappa(u,v)$ kann in einem Tag von einem 
Zwischenlager (oder Fabrik) zu einem anderen anderen 
(oder Verkaufstelle) transportiert werden. Unterschiedliche Kosten
f\"ur Transportwege treten bei dem Benutzten von verschiedenen 
Transportmitteln, z.B. LKW, Zug oder Flugzeug, auf. 

Weitere Anwendungsgebiete sind mit der aus der Graphentheorie 
bekannten \"Aquivalenz von maximalen F\"ussen und minimalen S
Schnitten zu entdecken. Ein \emph{minmaler Schnitte} f\"ur 
zwei Knoten $s,t$ eines Graphen $G = (V,E)$ is eine Teilmenge
 $S \subseteq E$ der Kanten mit minimaler Gr\"o\ss e, 
so dass nach Entfernen von $S$ aus $G$ kein Weg von $s$ nach $t$ 
mehr existiert. Wenn wir uns $s$ als das Versorgungsdepot 
und $t$ als Frontlinie eines milit\"arischen Gegners vorstellen, 
und $G$ das zugrunde liegende Transportnetzwerk ist, kann 
ein minimaler Schnitt in $G$ als der effizienteste Angriffsplan
interpretiert werden, um die Lieferung von Nachsch\"uben aus 
dem Versorgungsdepot and die Front zu unterbinden.

Als drittes Beispiel sei die Paarung von m\"oglichen Partner 
einer Partnerb\"oerse genannt. Gegeben sei eine Menge $A$ von 
Frauen und eine Menge $B$ von M\"annern. 
Durch soziologische Pr\"aferenzen von Frauen und  M\"anner 
ergeben sich potentielle Paarungen $E \subseteq A \times B$
zwischen den beiden Gruppen
\footnote{
Es ist zu bermerken das dieses Beispiel in Bezug auf Diversit\"at
ungeeignet ist und dass in der realen Welt gleichgeschlechtliche 
Beziehungen existieren. Da jedoch die hier erl\"auterten 
Algorithmen nur mit bipartiten Graphe korrekt funktionieren wurde
dieses ansonsten sehr anschauliche Beispiel gew\"ahlt.
}.
Der Betreiber einer Partnerb\"orse m\"ochte nun m\"oglichst viele
solcher potentiellen Paare mit einander zusammenbringen.
Dabei soll kein Partner in mehreren Paarungen auftreten. 
Das algorithmische Problem hierbei das Finden eines 
\emph{maximalen Matchings}, d.h. eine Kantenmenge 
$M \subset E$ des bipartiten Graphen $G = (A \cup B, E)$ mit 
maximaler Gr\"o\ss e, so dass der Teilgraph von $G$ der nur 
die Kanten $M$ besitzt Maximalgrad $1$ hat. Ein maximales Matching
kann gefunden werden in dem zu $G$ eine k\"unstliche Quelle $s$, 
die eine gerichtete Kante zu jedem Knoten aus $A$ erh\"alt, 
und eine Senke $t$, zu der eine eine gerichtete Kante von jedem 
Knoten in $B$ f\"uhrt, hinzugef\"ugt wird. Weiterhin wird jede
Kante in $E$ von $A$ nach $B$ gerichtet und wir geben jeder 
Kante Kapazit\"at eins. Es ist leicht zu sehen, 
dass ein Maximaler Fluss in diesem Graphen einem Maximalen 
Matching in $G$ entspricht: $M$ entspricht allen Kanten von $A$
nach $B$ mit Fluss $1$.


\end{document}

